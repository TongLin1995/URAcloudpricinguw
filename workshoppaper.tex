
\documentclass[11pt]{article}

\usepackage{times}
\usepackage{amsmath}
\setlength{\textwidth}{6in}
\setlength{\textheight}{8.75in}
\setlength{\oddsidemargin}{.3in}
\setlength{\topmargin}{-0.1in}
\setlength{\baselineskip}{14pt}

\pagestyle{headings}


\begin{document}

\begin{center}

{\large\bf Optimal Bid Strategy For Each Bidder\\ To Bid In Spot-Pricing Market}

\vspace{12pt}

{\em Tong Lin}

{\em t42lin@uwaterloo.ca}

{\em University of Waterloo}
\end{center}

$\ $

\section{Abstract}

With the introduction of cloud computing, more and more companies choose to use remote cloud computing---saving up-front hardware payment and pay for what they use. Many cloud provider including Amazon(Amazon EC2) developed different service charging standard in order to meet the demand for different user. These days, the most prevalent pricing methods are: on-demand pricing, reserved pricing and spot-pricing. With these three different pricing, cloud user can choose among them according to their needs. Obviously, the pros of these three method respectively are: pay for what you get, reserve instances to get a significant discount, pay less than what you get but have to bear the risk that the jobs are interrupted. 

When bidding the cloud, every bidder will come across with this problem: what is the optimal bidding strategy for me? Many scholars and researchers have published papers in this topic, trying to find the optimal bidding strategy. However, since each user has different needs when using cloud, there will never be an optimal strategy that can be applied to every user. In particularly, this article take each bidder's "true value"(i.e. the revenue he will have if he pay nothing for the service) into account. We will treat spot market as an sealed-bid N+1 auction market with several assumptions including Spot-pricing market is a supply-demand market, even though there are some evidences that providers like Amazon are manually hiding instances to reduce demand. The contribution of this article will give an full set of algorithms of how to bid, when to bid and which price to choose.


\section{Introduction}

With the increasing innovation of computers, computing becomes much important than before. Individual programmers who want to perform large computations will need helps from cloud computing. By definition of Wikipedia, "Cloud computing, also on-demand computing, is a kind of Internet-based computing that provides shared processing resources and data to computers and other devices on demand. It is a model for enabling ubiquitous, on-demand access to a shared pool of configurable computing resources." We can see many real world application of cloud.

For the terminology we will use, the provider means cloud provider like Amazon EC2. Bidders means people who submit their bid in 

\section{Background and inspirations}

The model I used was inspired by auction markets. I used the fact that the spot pricing market is similar to N+1 sealed-bid auction market.

\section{Model}

For bidders, they will first of all observe the historical mean spot price to determine which bid value will he choose to use. After all bidder submit their bid, the provider will use their algorithm to determine a spot price. All bids with bid price over spot price got accepted. Now, consider a model with following assumptions. Each bidder only bid one unit of cloud. So they will only submit one bid price and the outcome for them is accepted or not. Ignore the unused time for instances, historically around 70 percent of the whole successful bidder will only use less than 33 percent of the whole time interval. But if we focus on how much work they finished using cloud rather than how many time they used we can validate a variable in our model.

Consider a auction market, everybody will have their own desired value(i.e. the maximum price they are willing to pay). Then the bidder will choose their own strategy to determine a price based on their desired value. Now for cloud bidding market. We define "True Value" of a bidder as the maximum of price he is willing to pay for unit service, noted by $v_i$ for user i. Define $\sigma_i v_i$ as the lost of user i when the user didn't bid it successfully. And we define $s(t)$ as spot price at time t. $b(t)$ is the bid price the user choose. A function $F()$ as the choosing strategy function for user i, thus, $b(t)=F(v(t))$. Note that we have $F$ continuously differentiable and increasing. Define on-demand price as $d(t)$ and we assume $d$ is constant for the time interval we are interested in.

\subsection{Propositions}
Claim: 


\section{Single bidder's optimal bidding strategy}
For users, if choose on-demand pricing, the expected payoff for him is $Ed_i=1(v_i-d)$. If choose spot pricing, the expected payoff for him is $Es_i=(1-P(F(v_i)\leq s(t)))(v_i-s(t))-P(F(v_i)\leq s(t))\sigma v_i$. Now if we want the user to choose spot pricing instead of on demand pricing, we will need$G_i=Es_i-Ed_i\geq 0$

\begin{equation} 
\begin{split}
G_i & =(1-P(F(v_i)\leq s(t)))-P(F(v_i)\leq s(t))\sigma v_i-v_i+d\\
& =v_i-s(t)-P(F(v_i)\leq s(t))v_i+P(F(v_i)\leq s(t))s(t)-P(F(v_i)\leq s(t))\sigma v_i-v_i+d\\
& =-d-P(F(v_i)\leq s(t))v_i+P(F(v_i)\leq s(t))s(t)-P(F(v_i)\leq s(t))\sigma v_i+d\\
& =-P(F(v_i)\leq s(t))(v_i+v_i\sigma-s(t))+d-s(t)
\end{split}
\end{equation}

Now since we have $v_i > s(t)$(otherwise we won't consider to use cloud) we can see that the best way to maximize your payoff is to minimize your probability to be an unsuccessful bidder.

%\section{Experiment of applying our strategy to historical data} 

%\section{Conclusion}

%\section{References}



\end{document}


